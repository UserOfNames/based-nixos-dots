\documentclass{article}

% \usepackage{bibtex}

\author{Zachary Dierberger}
\title{My first {\LaTeX} document}
\date{Sometime}

\begin{document}

\maketitle

\section{Introduction}

This is some text. This is a new line in my editor. However, note that it isn't
a newline in LaTeX. This is because...

To make a new line, you must use two enters.

In section 4, we discuss labels and references. This will make sense soon.

In section \ref{labref}, we discuss labels and references. This will make sense
soon.



\section{Formatting}

Sections are automatically numbered. These numbers update when the document is compiled.

\subsection{Subsection}

These are automatically numbered as well!

\subsection{Custom formatting}

If you don't like the default format, you can change it later. It's best to
write the outline first, then format later. More on that later, though.

\subsection{Text formatting}

This is normal text. \textbf{This is bold text.} \textit{This is italic text
use this for anything that's always italicized.} \emph{This is emphatic text.
Use this for stuff that's just emphasized, but not necessarily proper. This is
because emphatic can be changed from italic to other methods.}
\underline{Underlined.}

"This is in quotations. Note that the opening quote is facing the wrong way. To
make them correct, use two graves and two single quotes (or singles for single
quotes!):"

``This is in proper quotation marks.''

\section{Lists} 

\begin{enumerate}
	\item Enumerate creates ordered lists.
	\item Element 1.
		\begin{enumerate}
			\item Subelement 1
		\end{enumerate}
	\item Element 2.
\end{enumerate}

\begin{itemize}
	\item Itemize creates unordered lists.
	\item Element 1.
		\begin{itemize}
			\item Subelement 1
		\end{itemize}
	\item Element 2.
\end{itemize}

\section{Labels and references\label{labref}}

Say you write a long document and you need to reuse or elaborate on certain
parts of it at a later point. Basically, you're taking parts of documents and
putting them in other documents. As we've seen, Latex redoes numbering for us.
But what about referencing specific items in text?

For example, consider the earlier "Introduction" section. It refers to this as
"section 4," but if we move this around later, it won't be section 4 anymore.
Labels and references let you manage numbers like this automatically.

As you can see in the section heading, we created a label. This label is
abstract and does not show up in the document after compilation. We can then
use a reference to this label wherever, as you can see in introduction. If we
were to move this section around, the plain line would remain fixed but the
reference line would update accordingly.

\section{Bibliography}

You can construct a long-term bibliography containing various sources, which
you may then cite in any other document as you please. Some examples on what
entries in this may look like:

@book\{tag,\\
	author = "Doe, John",\\
	title = "Foo Bar",\\
	year = "2000",\\
	publisher = "Bar Publishing"\\
\}

There are different packages for bibliography management. One is biber. You can
see where I imported the biblatex package in the preamble. Adding square brackets
before the package name lets you specify parameters, such as backend=biber.
To import the bibliography, use addbibresource\{filename.bib\}.

Now, whenever you call the \emph{tag}, 

\end{document}
